% \documentclass[xcolor=pdflatex,dvipsnames,table]{beamer}
% \usepackage{epsfig,graphicx}
% \usepackage{palatino}
% \usepackage{fancybox}
% \usepackage{relsize}
% \usepackage[procnames]{listings}
% 
% \input{../style/scala.tex}
% % "define" Stanza
\usepackage[T1]{fontenc}  
\usepackage[scaled=0.82]{beramono}  
\usepackage{microtype} 

\sbox0{\small\ttfamily A}
\edef\mybasewidth{\the\wd0 }

\lstdefinelanguage{stanza}{
  morekeywords={circuit, defclass,defmodule,defbundle,definterface,defpackage,defn,%
    do,else,public,false,finally,%
    for,if,import,inherit,inst,match,%
    map,new,node,object,override,package,%
    super,this,throw,true,try,%
    type,val,var,when,while%,
    % yield,UInt,Bool,Bits,SInt
    },
  sensitive=true,
  morecomment=[l]{;;},
  morestring=[b]",
  morestring=[b]',
  morestring=[b]"""
}

\usepackage{color}
\definecolor{dkgreen}{rgb}{0,0.6,0}
\definecolor{gray}{rgb}{0.5,0.5,0.5}
\definecolor{mauve}{rgb}{0.58,0,0.82}

% Default settings for code listings
\lstset{frame=tb,
  language=stanza,
  aboveskip=3mm,
  belowskip=3mm,
  showstringspaces=false,
  columns=fixed, % basewidth=\mybasewidth,
  basicstyle={\small\ttfamily},
  numbers=none,
  numberstyle=\footnotesize\color{gray},
  % identifierstyle=\color{red},
  keywordstyle=\color{blue},
  commentstyle=\color{dkgreen},
  stringstyle=\color{mauve},
  frame=single,
  breaklines=true,
  breakatwhitespace=true,
  procnamekeys={input,output,wire, mem, reg, node, defn, val, var, defclass, definterface, defbundle, defmodule, defpackage},
  procnamestyle=\ttfamily\color{red},
  tabsize=2
}

\lstnewenvironment{stanza}[1][]
{\lstset{language=stanza,#1}}
{}

% "define" Stanza
\usepackage[T1]{fontenc}  
\usepackage[scaled=0.82]{beramono}  
\usepackage{microtype} 

\sbox0{\small\ttfamily A}
\edef\mybasewidth{\the\wd0 }

\lstdefinelanguage{stanza}{
  morekeywords={circuit, defclass,defmodule,defbundle,definterface,defpackage,defn,%
    do,else,public,false,finally,%
    for,if,import,inherit,input,inst,match,%
    map,mem,new,node,object,override,output,package,%
    reg,super,this,throw,true,try,%
    type,val,var,when,while,wire%,%
    % yield,UInt,Bool,Bits,SInt
    },
  sensitive=true,
  morecomment=[l]{;;},
  morestring=[b]",
  morestring=[b]',
  morestring=[b]"""
}

\usepackage{color}
\definecolor{dkgreen}{rgb}{0,0.6,0}
\definecolor{gray}{rgb}{0.5,0.5,0.5}
\definecolor{mauve}{rgb}{0.58,0,0.82}

% Default settings for code listings
\lstset{frame=tb,
  language=stanza,
  aboveskip=3mm,
  belowskip=3mm,
  showstringspaces=false,
  columns=fixed, % basewidth=\mybasewidth,
  basicstyle={\small\ttfamily},
  numbers=none,
  numberstyle=\footnotesize\color{gray},
  % identifierstyle=\color{red},
  keywordstyle=\color{blue},
  commentstyle=\color{dkgreen},
  stringstyle=\color{mauve},
  frame=single,
  breaklines=true,
  breakatwhitespace=true,
  procnamekeys={input,output,wire, mem, reg, node, defn, val, var, defclass, definterface, defbundle, defmodule, defpackage},
  procnamestyle=\ttfamily\color{red},
  tabsize=2
}



% \input{../style/talk.tex}
% 
% 
% \begin{document}

\begin{frame}[fragile]{Stanza* Essence}
\begin{itemize}
\item best of scripting and production languages
\begin{itemize}
\item easy to understand and powerful to use
\item gradual types -> easy parameteric types
\end{itemize}
\item simple orthogonal concepts
\begin{itemize}
\item functions, objects, pipes, and namespace separated
\item use concepts in unlimited ways -- serendipity
\item entire language including optimizing native compiler in 20K LOC
\end{itemize}
\item powerful macros for conventional syntax
\begin{itemize}
\item almost entire stanza syntax written as macros
\item better DSL hosting language
\end{itemize}
\end{itemize}
\vspace{0.25cm}
\begin{center}
\includegraphics[height=0.2\textheight]{figs/patrick.jpg} \\
* developed by Patrick Li @ EECS Berkeley
\end{center}
\end{frame}

\begin{frame}[fragile]{Stanza by Contrast}
\begin{itemize}
\item like Python but
\begin{itemize}
\item types and on and on and on ...
\end{itemize}
\item like Scala but
\begin{itemize}
\item thinner -- native compiler + runtime in 20K LOC
\item simpler -- fewer base concepts
\item more separated -- orthogonal functions / objects
\end{itemize}
\item like Clojure but
\begin{itemize}
\item conventional syntax -- love the parens but ...
\item thinner -- native
\item more powerful macro system -- not just name macros
\end{itemize}
\item like Dylan but
\begin{itemize}
\item improved gradual types -- parameteric types
\item better multimethod namespaces -- fewer name clashes
\item more powerful macros -- syntax written in it
\item has pipes -- generalized control flow mechanism
\end{itemize}
\end{itemize}
\end{frame}

% \end{document}
